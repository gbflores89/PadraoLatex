%
%
%
\chapter{Normas do padrão de dissertação/tese/qualify} \label{chap:normas}


\emph{Neste capítulo estão sumarizadas as normas para o padrão de dissertação/tese/qualify
do Programa de Pós-Graduação em Engenharia Química da Universidade Federal do Rio Grande
do Sul (PPGEQ--UFRGS).
O presente documento está de acorodo com estas normas e pode ser utilizado como guia.
Nas seções abaixo são explicitados os padrões assumidos.}

\section{Dimensões e formato geral}

O documento deverá utilizar dimensões padrão A4.
As margens laterais serão de acordo com ABNT, com 3~cm na esquerda e 2~cm direita.
Isto fará com que o documento, após encadernação, fique com 2~cm de margem em
ambos os lados.
O documento deverá ser produzido para a impressão frente e verso.

Ainda de acordo com a ABNT, as margens superior e inferior seriam de 3 e 2~cm, respectivamente.
Entretanto, isto significa em uma perda considerável de espaço, além de uma assimetria.
Fica determinado que as margens superior e inferior serão de 2~cm.

\section{Espaçamento entre linhas e tamanho de fonte}

Seguindo recomendações tipográficas, o tamanho da fonte deve ser tal que o número de caracteres
por linha varie entre 60 e 70. Isto é obtido usualmente com uma fonte tamanho 12.
O espaçamento entre linhas deverá ser de 1,5 no corpo do texto.
Especialmente na seção de referências bibliográficas, o espaçamento será simples.
O corpo do texto deve ser em fonte tipo Times ou similar.

\section{Numeração}

Todo o início de seção (Listagem, Capítulo, Apêndice, etc.) deverá iniciar em página impar,
fazendo com que fique sempre na página à direita do leitor.
Poderá haver necessidade de deixar uma página em branco para respeitar esta regra.
No padrão em \LaTeX{} este comportamento é automatizado. 

As folhas de rosto, listagens, agradecimentos, etc. são numeradas em números romanos
centralizados no rodapé. Estas páginas não contém cabeçalho e devem
conter exatamente as páginas exemplificadas no início deste documento.

A contagem de páginas do documento (com números arábicos) inicia-se na primeira página do Capítulo~1.

A primeira página de cada capítulo não deverá conter cabeçalho, apenas o número da página
centralizada no rodapé.
A páginas do corpo do documento deverão conter um cabeçalho.
Páginas pares deverão apresentar o número da página no extremo esquerdo e a referência ao
capítulo atual no extremo direito.
As páginas ímpares deverão apresentar a referência à seção no extremo esquerdo e o número
da página no extremo direito.

Cada capítulo deverá iniciar com ``Capítulo X'' e, uma linha abaixo, deverá
ser apresentado o título do capítulo.
Também poderá estar presente uma pequena sinopse
em itálico, antes da primeira seção do capítulo.
As seções deverão ser numerádas até 3 níveis. Por exemplo 1.1, 1.2, 1.2.3.
Um quarto nível deverá ser evitado.
Se necessário um quarto nível, este não deverá ser numerado.

Os títulos das seções e capítulos deverão ser em fonte tipo Arial e negrito.
Para o segundo nível de numeração (p. ex. 3.1) deverá ser utilizanda uma fonte maior, tamanho 14.
Para o terceiro nível de numeração (p. ex. 3.2.2) dererá ser utilizado apenas o negrito e
fonte tamanho nornal.
Os textos nos cabeçalhos também em fonte tipo Arial.

O sumário deverá apresentar todas as seções do documento, iniciando com as páginas
das listagens de figuras, tabelas, etc.
O sumário deverá apresentar até o terceiro nível de numeração, ex. 1.2.3.
No sumário, os títulos de capítulos devem estar em fonte tipo Arial e em negrito,
os demais em fonte simples tipo Times.

\section{Ficha catalográfica}

Uma ficha catalográfica deverá estar presente no verso da primeira página do
documento (não no verso da capa -- o que equivale a terceira página do
presente documento).

Verificar se a ficha catalográfica está exatamente como a produzida
pelo sistema disponível em \url{http://sabi.ufrgs.br/servicos/publicoBC/ficha.php}.
O padrão em \LaTeX{} automatiza a produção da ficha catalográfica.

\section{Citações}

As citações devem seguir o padrão autor--ano da ABNT. Trabalhos com mais de dois
autores deverão ser abreviados com \emph{et al.}.
Porém, na listagem de referências, todos os autores devem ser apresentados.

Citações no corpo do texto, devem apresentar o nome dos autores com o ano entre parênteses,
da seguinte forma: no trabalho de \citeonline{Soares:2003}
o assunto X foi estudado.
Citações em final de frase devem ser apresentar os nomes em maiúsculas seguido do ano,
da seguinte forma: tal assunto não é abordado na literatura \cite{Soares:2003}.
